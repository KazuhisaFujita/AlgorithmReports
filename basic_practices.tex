%%%%%%%%%%%%%%%%%%%%%%%%%%%%%
%Since : 2015/03/27
%Update: <2015/03/27>
% -*- coding: iso-2022-jp -*-
%%%%%%%%%%%%%%%%%%%%%%%%%%%%%
\documentclass[a4j, 11pt, fleqn]{jsarticle}
\usepackage{bm}
\usepackage{listings}
% \setlength{\columnseprule}{0.4pt}
% \setlength{\oddsidemargin}{-5mm}
 \setlength{\topmargin}{-20mm}
% \setlength{\textwidth}{325mm}
 \setlength{\textheight}{230mm}

\begin{document}
{\LARGE \bf アルゴリズムとデータ構造 演習 2016/4/11,18}

\begin{enumerate}

  \item Hello Worldを表示させよ。
    \item 2つの変数$x$、$y$の大小を比較し、大きい方の数を表示するプログラムを作
          成せよ。
    \item 変数$x$の絶対値を表示するプログラムを作成せよ。
    \item FizzBuzz問題を解くプログラムを作成せよ。(1からNまでの整数を表示するが、3で割り切れる倍は"Fizz"、5で割り切れる場合は"Buzz"、3と5で割り切れる場合は"FizzBuzz"を表示する。)
    \item 九九表を表示するプログラムを作成せよ。ただし、for文を使うこと。
    \item 釣り銭の支払い紙幣と硬貨を計算するプログラムを作成せよ。ただし、釣り銭
          は日本円とし、紙幣と硬貨の枚数と個数の総和は最小とせよ。
    \item 変数の絶対値を返す関数abを作れ。
    \item 人間とコンピュータがじゃんけんできるプログラムを作成せよ。ただし、コンピュータはランダムに手をだし、人間が勝つまでじゃんけんを繰り返すようにせよ。
    \item 要素が100個ある配列に乱数を入れ,その平均を求めるプログラムを作成せよ.
    \item 要素が100個ある配列に乱数を入れ,その100個の数の最大値と最小値を求めるプログラムを作成せよ.
    \item 2次元デカルト座標$(x, y)$を極座標$(r, \theta)$に変換するプログラムを作成せよ.
    \item 3つの線分の長さ$a, b, c$を入力した場合,その線分を使って三角形が作れるかどうかを判定するプログラムを作成せよ.ただし,判定部分は関数化せよ.
    \item 100個の乱数をファイルに出力するプログラムを作れ.
    \item 前問で作成したプログラムにより作成されたファイルに記載された数値のうち,5の倍数のみ画面に出力するプログラムを作れ.
    \item コマンドライン引数として2つの整数を与えると,その2つの整数値のうち大きい方を画面に出力するようなプログラムを作成せよ.
    \item 速度$\bm v = (v_x, x_y)$でボールを投げた.時刻$t$におけるボールの場所を求めるプログラムを作れ.コーディングの際,場所を計算するコードは関数化せよ.
    ボールを投げた場所を原点とし,重力加速度は$\bm a = (0, -9.8)$ m/s$^2$で計算すること.投げる方向に制限はない.

\end{enumerate}



\end{document}
